% Options for packages loaded elsewhere
\PassOptionsToPackage{unicode}{hyperref}
\PassOptionsToPackage{hyphens}{url}
%
\documentclass[
]{article}
\usepackage{amsmath,amssymb}
\usepackage{iftex}
\ifPDFTeX
  \usepackage[T1]{fontenc}
  \usepackage[utf8]{inputenc}
  \usepackage{textcomp} % provide euro and other symbols
\else % if luatex or xetex
  \usepackage{unicode-math} % this also loads fontspec
  \defaultfontfeatures{Scale=MatchLowercase}
  \defaultfontfeatures[\rmfamily]{Ligatures=TeX,Scale=1}
\fi
\usepackage{lmodern}
\ifPDFTeX\else
  % xetex/luatex font selection
\fi
% Use upquote if available, for straight quotes in verbatim environments
\IfFileExists{upquote.sty}{\usepackage{upquote}}{}
\IfFileExists{microtype.sty}{% use microtype if available
  \usepackage[]{microtype}
  \UseMicrotypeSet[protrusion]{basicmath} % disable protrusion for tt fonts
}{}
\makeatletter
\@ifundefined{KOMAClassName}{% if non-KOMA class
  \IfFileExists{parskip.sty}{%
    \usepackage{parskip}
  }{% else
    \setlength{\parindent}{0pt}
    \setlength{\parskip}{6pt plus 2pt minus 1pt}}
}{% if KOMA class
  \KOMAoptions{parskip=half}}
\makeatother
\usepackage{xcolor}
\usepackage[margin=1in]{geometry}
\usepackage{color}
\usepackage{fancyvrb}
\newcommand{\VerbBar}{|}
\newcommand{\VERB}{\Verb[commandchars=\\\{\}]}
\DefineVerbatimEnvironment{Highlighting}{Verbatim}{commandchars=\\\{\}}
% Add ',fontsize=\small' for more characters per line
\usepackage{framed}
\definecolor{shadecolor}{RGB}{248,248,248}
\newenvironment{Shaded}{\begin{snugshade}}{\end{snugshade}}
\newcommand{\AlertTok}[1]{\textcolor[rgb]{0.94,0.16,0.16}{#1}}
\newcommand{\AnnotationTok}[1]{\textcolor[rgb]{0.56,0.35,0.01}{\textbf{\textit{#1}}}}
\newcommand{\AttributeTok}[1]{\textcolor[rgb]{0.13,0.29,0.53}{#1}}
\newcommand{\BaseNTok}[1]{\textcolor[rgb]{0.00,0.00,0.81}{#1}}
\newcommand{\BuiltInTok}[1]{#1}
\newcommand{\CharTok}[1]{\textcolor[rgb]{0.31,0.60,0.02}{#1}}
\newcommand{\CommentTok}[1]{\textcolor[rgb]{0.56,0.35,0.01}{\textit{#1}}}
\newcommand{\CommentVarTok}[1]{\textcolor[rgb]{0.56,0.35,0.01}{\textbf{\textit{#1}}}}
\newcommand{\ConstantTok}[1]{\textcolor[rgb]{0.56,0.35,0.01}{#1}}
\newcommand{\ControlFlowTok}[1]{\textcolor[rgb]{0.13,0.29,0.53}{\textbf{#1}}}
\newcommand{\DataTypeTok}[1]{\textcolor[rgb]{0.13,0.29,0.53}{#1}}
\newcommand{\DecValTok}[1]{\textcolor[rgb]{0.00,0.00,0.81}{#1}}
\newcommand{\DocumentationTok}[1]{\textcolor[rgb]{0.56,0.35,0.01}{\textbf{\textit{#1}}}}
\newcommand{\ErrorTok}[1]{\textcolor[rgb]{0.64,0.00,0.00}{\textbf{#1}}}
\newcommand{\ExtensionTok}[1]{#1}
\newcommand{\FloatTok}[1]{\textcolor[rgb]{0.00,0.00,0.81}{#1}}
\newcommand{\FunctionTok}[1]{\textcolor[rgb]{0.13,0.29,0.53}{\textbf{#1}}}
\newcommand{\ImportTok}[1]{#1}
\newcommand{\InformationTok}[1]{\textcolor[rgb]{0.56,0.35,0.01}{\textbf{\textit{#1}}}}
\newcommand{\KeywordTok}[1]{\textcolor[rgb]{0.13,0.29,0.53}{\textbf{#1}}}
\newcommand{\NormalTok}[1]{#1}
\newcommand{\OperatorTok}[1]{\textcolor[rgb]{0.81,0.36,0.00}{\textbf{#1}}}
\newcommand{\OtherTok}[1]{\textcolor[rgb]{0.56,0.35,0.01}{#1}}
\newcommand{\PreprocessorTok}[1]{\textcolor[rgb]{0.56,0.35,0.01}{\textit{#1}}}
\newcommand{\RegionMarkerTok}[1]{#1}
\newcommand{\SpecialCharTok}[1]{\textcolor[rgb]{0.81,0.36,0.00}{\textbf{#1}}}
\newcommand{\SpecialStringTok}[1]{\textcolor[rgb]{0.31,0.60,0.02}{#1}}
\newcommand{\StringTok}[1]{\textcolor[rgb]{0.31,0.60,0.02}{#1}}
\newcommand{\VariableTok}[1]{\textcolor[rgb]{0.00,0.00,0.00}{#1}}
\newcommand{\VerbatimStringTok}[1]{\textcolor[rgb]{0.31,0.60,0.02}{#1}}
\newcommand{\WarningTok}[1]{\textcolor[rgb]{0.56,0.35,0.01}{\textbf{\textit{#1}}}}
\usepackage{graphicx}
\makeatletter
\def\maxwidth{\ifdim\Gin@nat@width>\linewidth\linewidth\else\Gin@nat@width\fi}
\def\maxheight{\ifdim\Gin@nat@height>\textheight\textheight\else\Gin@nat@height\fi}
\makeatother
% Scale images if necessary, so that they will not overflow the page
% margins by default, and it is still possible to overwrite the defaults
% using explicit options in \includegraphics[width, height, ...]{}
\setkeys{Gin}{width=\maxwidth,height=\maxheight,keepaspectratio}
% Set default figure placement to htbp
\makeatletter
\def\fps@figure{htbp}
\makeatother
\setlength{\emergencystretch}{3em} % prevent overfull lines
\providecommand{\tightlist}{%
  \setlength{\itemsep}{0pt}\setlength{\parskip}{0pt}}
\setcounter{secnumdepth}{-\maxdimen} % remove section numbering
\usepackage{booktabs}
\usepackage{caption}
\usepackage{longtable}
\usepackage{colortbl}
\usepackage{array}
\usepackage{anyfontsize}
\usepackage{multirow}
\ifLuaTeX
  \usepackage{selnolig}  % disable illegal ligatures
\fi
\usepackage{bookmark}
\IfFileExists{xurl.sty}{\usepackage{xurl}}{} % add URL line breaks if available
\urlstyle{same}
\hypersetup{
  pdftitle={formR.R},
  pdfauthor={Mr ELYSEE},
  hidelinks,
  pdfcreator={LaTeX via pandoc}}

\title{formR.R}
\author{Mr ELYSEE}
\date{2025-06-18}

\begin{document}
\maketitle

\begin{Shaded}
\begin{Highlighting}[]
\FunctionTok{options}\NormalTok{(}\AttributeTok{repos =} \FunctionTok{c}\NormalTok{(}\AttributeTok{CRAN =} \StringTok{"https://cloud.r{-}project.org/"}\NormalTok{))}
\FunctionTok{install.packages}\NormalTok{(}\StringTok{"tidyverse"}\NormalTok{)}
\end{Highlighting}
\end{Shaded}

\begin{verbatim}
## le package 'tidyverse' a été décompressé et les sommes MD5 ont été vérifiées avec succés
## 
## Les packages binaires téléchargés sont dans
##  C:\Users\Mr ELYSEE\AppData\Local\Temp\RtmpGuCpNQ\downloaded_packages
\end{verbatim}

\begin{Shaded}
\begin{Highlighting}[]
\FunctionTok{install.packages}\NormalTok{(}\StringTok{"ggplot2"}\NormalTok{)}
\end{Highlighting}
\end{Shaded}

\begin{verbatim}
## le package 'ggplot2' a été décompressé et les sommes MD5 ont été vérifiées avec succés
## 
## Les packages binaires téléchargés sont dans
##  C:\Users\Mr ELYSEE\AppData\Local\Temp\RtmpGuCpNQ\downloaded_packages
\end{verbatim}

\begin{Shaded}
\begin{Highlighting}[]
\FunctionTok{install.packages}\NormalTok{(}\StringTok{"labelled"}\NormalTok{)}
\end{Highlighting}
\end{Shaded}

\begin{verbatim}
## le package 'labelled' a été décompressé et les sommes MD5 ont été vérifiées avec succés
## 
## Les packages binaires téléchargés sont dans
##  C:\Users\Mr ELYSEE\AppData\Local\Temp\RtmpGuCpNQ\downloaded_packages
\end{verbatim}

\begin{Shaded}
\begin{Highlighting}[]
\FunctionTok{install.packages}\NormalTok{(}\StringTok{"GGally"}\NormalTok{)}
\end{Highlighting}
\end{Shaded}

\begin{verbatim}
## le package 'GGally' a été décompressé et les sommes MD5 ont été vérifiées avec succés
## 
## Les packages binaires téléchargés sont dans
##  C:\Users\Mr ELYSEE\AppData\Local\Temp\RtmpGuCpNQ\downloaded_packages
\end{verbatim}

\begin{Shaded}
\begin{Highlighting}[]
\FunctionTok{install.packages}\NormalTok{(}\StringTok{"lubridate"}\NormalTok{)}
\end{Highlighting}
\end{Shaded}

\begin{verbatim}
## le package 'lubridate' a été décompressé et les sommes MD5 ont été vérifiées avec succés
## 
## Les packages binaires téléchargés sont dans
##  C:\Users\Mr ELYSEE\AppData\Local\Temp\RtmpGuCpNQ\downloaded_packages
\end{verbatim}

\begin{Shaded}
\begin{Highlighting}[]
\FunctionTok{install.packages}\NormalTok{(}\StringTok{"forcats"}\NormalTok{)}
\end{Highlighting}
\end{Shaded}

\begin{verbatim}
## le package 'forcats' a été décompressé et les sommes MD5 ont été vérifiées avec succés
## 
## Les packages binaires téléchargés sont dans
##  C:\Users\Mr ELYSEE\AppData\Local\Temp\RtmpGuCpNQ\downloaded_packages
\end{verbatim}

\begin{Shaded}
\begin{Highlighting}[]
\FunctionTok{install.packages}\NormalTok{(}\StringTok{"gtsummary"}\NormalTok{)}
\end{Highlighting}
\end{Shaded}

\begin{verbatim}
## installation de la dépendance 'cli'
\end{verbatim}

\begin{verbatim}
## le package 'cli' a été décompressé et les sommes MD5 ont été vérifiées avec succés
\end{verbatim}

\begin{verbatim}
## Warning: impossible de supprimer l'installation précédente du package 'cli'
\end{verbatim}

\begin{verbatim}
## Warning in file.copy(savedcopy, lib, recursive = TRUE): problème lors de la
## copie de D:\R\R-4.4.1\library\00LOCK\cli\libs\x64\cli.dll vers
## D:\R\R-4.4.1\library\cli\libs\x64\cli.dll : Permission denied
\end{verbatim}

\begin{verbatim}
## Warning: 'cli' restauré
\end{verbatim}

\begin{verbatim}
## le package 'gtsummary' a été décompressé et les sommes MD5 ont été vérifiées avec succés
## 
## Les packages binaires téléchargés sont dans
##  C:\Users\Mr ELYSEE\AppData\Local\Temp\RtmpGuCpNQ\downloaded_packages
\end{verbatim}

\begin{Shaded}
\begin{Highlighting}[]
\FunctionTok{library}\NormalTok{(tidyverse)}
\end{Highlighting}
\end{Shaded}

\begin{verbatim}
## Warning: le package 'tidyverse' a été compilé avec la version R 4.4.3
\end{verbatim}

\begin{verbatim}
## Warning: le package 'ggplot2' a été compilé avec la version R 4.4.3
\end{verbatim}

\begin{verbatim}
## Warning: le package 'purrr' a été compilé avec la version R 4.4.3
\end{verbatim}

\begin{verbatim}
## Warning: le package 'forcats' a été compilé avec la version R 4.4.3
\end{verbatim}

\begin{verbatim}
## Warning: le package 'lubridate' a été compilé avec la version R 4.4.3
\end{verbatim}

\begin{verbatim}
## -- Attaching core tidyverse packages ------------------------ tidyverse 2.0.0 --
## v dplyr     1.1.4     v readr     2.1.5
## v forcats   1.0.0     v stringr   1.5.1
## v ggplot2   3.5.2     v tibble    3.2.1
## v lubridate 1.9.4     v tidyr     1.3.1
## v purrr     1.0.4
\end{verbatim}

\begin{verbatim}
## -- Conflicts ------------------------------------------ tidyverse_conflicts() --
## x dplyr::filter() masks stats::filter()
## x dplyr::lag()    masks stats::lag()
## i Use the conflicted package (<http://conflicted.r-lib.org/>) to force all conflicts to become errors
\end{verbatim}

\begin{Shaded}
\begin{Highlighting}[]
\FunctionTok{library}\NormalTok{(car)}
\end{Highlighting}
\end{Shaded}

\begin{verbatim}
## Warning: le package 'car' a été compilé avec la version R 4.4.3
\end{verbatim}

\begin{verbatim}
## Le chargement a nécessité le package : carData
\end{verbatim}

\begin{verbatim}
## Warning: le package 'carData' a été compilé avec la version R 4.4.2
\end{verbatim}

\begin{verbatim}
## 
## Attachement du package : 'car'
## 
## L'objet suivant est masqué depuis 'package:dplyr':
## 
##     recode
## 
## L'objet suivant est masqué depuis 'package:purrr':
## 
##     some
\end{verbatim}

\begin{Shaded}
\begin{Highlighting}[]
\FunctionTok{library}\NormalTok{(cardx)}
\end{Highlighting}
\end{Shaded}

\begin{verbatim}
## Warning: le package 'cardx' a été compilé avec la version R 4.4.3
\end{verbatim}

\begin{Shaded}
\begin{Highlighting}[]
\FunctionTok{library}\NormalTok{(ggplot2)}
\FunctionTok{library}\NormalTok{(labelled)}
\end{Highlighting}
\end{Shaded}

\begin{verbatim}
## Warning: le package 'labelled' a été compilé avec la version R 4.4.3
\end{verbatim}

\begin{Shaded}
\begin{Highlighting}[]
\FunctionTok{library}\NormalTok{(gtsummary)}
\end{Highlighting}
\end{Shaded}

\begin{verbatim}
## Warning: le package 'gtsummary' a été compilé avec la version R 4.4.3
\end{verbatim}

\begin{Shaded}
\begin{Highlighting}[]
\FunctionTok{library}\NormalTok{(ggstatsplot)}
\end{Highlighting}
\end{Shaded}

\begin{verbatim}
## Warning: le package 'ggstatsplot' a été compilé avec la version R 4.4.3
\end{verbatim}

\begin{verbatim}
## You can cite this package as:
##      Patil, I. (2021). Visualizations with statistical details: The 'ggstatsplot' approach.
##      Journal of Open Source Software, 6(61), 3167, doi:10.21105/joss.03167
\end{verbatim}

\begin{Shaded}
\begin{Highlighting}[]
\FunctionTok{library}\NormalTok{(Hmisc)}
\end{Highlighting}
\end{Shaded}

\begin{verbatim}
## Warning: le package 'Hmisc' a été compilé avec la version R 4.4.3
\end{verbatim}

\begin{verbatim}
## 
## Attachement du package : 'Hmisc'
## 
## Les objets suivants sont masqués depuis 'package:dplyr':
## 
##     src, summarize
## 
## Les objets suivants sont masqués depuis 'package:base':
## 
##     format.pval, units
\end{verbatim}

\begin{Shaded}
\begin{Highlighting}[]
\FunctionTok{data}\NormalTok{(}\StringTok{"trial"}\NormalTok{)}
\FunctionTok{theme\_gtsummary\_language}\NormalTok{(}\StringTok{"fr"}\NormalTok{,}\AttributeTok{decimal.mark =} \StringTok{","}\NormalTok{, }\AttributeTok{big.mark =} \StringTok{" "}\NormalTok{)}
\end{Highlighting}
\end{Shaded}

\begin{verbatim}
## Setting theme "language: fr"
\end{verbatim}

\begin{Shaded}
\begin{Highlighting}[]
\FunctionTok{View}\NormalTok{(trial)}
\NormalTok{trial}\SpecialCharTok{\%\textgreater{}\%}
  \FunctionTok{look\_for}\NormalTok{()}
\end{Highlighting}
\end{Shaded}

\begin{verbatim}
##  pos variable label                  col_type missing values
##  1   trt      Chemotherapy Treatment chr      0             
##  2   age      Age                    dbl      11            
##  3   marker   Marker Level (ng/mL)   dbl      10            
##  4   stage    T Stage                fct      0       T1    
##                                                       T2    
##                                                       T3    
##                                                       T4    
##  5   grade    Grade                  fct      0       I     
##                                                       II    
##                                                       III   
##  6   response Tumor Response         int      7             
##  7   death    Patient Died           int      0             
##  8   ttdeath  Months to Death/Censor dbl      0
\end{verbatim}

\begin{Shaded}
\begin{Highlighting}[]
\NormalTok{trial}\SpecialCharTok{|\textgreater{}}\FunctionTok{nrow}\NormalTok{()}
\end{Highlighting}
\end{Shaded}

\begin{verbatim}
## [1] 200
\end{verbatim}

\begin{Shaded}
\begin{Highlighting}[]
\FunctionTok{str}\NormalTok{(trial)}
\end{Highlighting}
\end{Shaded}

\begin{verbatim}
## tibble [200 x 8] (S3: tbl_df/tbl/data.frame)
##  $ trt     : chr [1:200] "Drug A" "Drug B" "Drug A" "Drug A" ...
##   ..- attr(*, "label")= chr "Chemotherapy Treatment"
##  $ age     : num [1:200] 23 9 31 NA 51 39 37 32 31 34 ...
##   ..- attr(*, "label")= chr "Age"
##  $ marker  : num [1:200] 0.16 1.107 0.277 2.067 2.767 ...
##   ..- attr(*, "label")= chr "Marker Level (ng/mL)"
##  $ stage   : Factor w/ 4 levels "T1","T2","T3",..: 1 2 1 3 4 4 1 1 1 3 ...
##   ..- attr(*, "label")= chr "T Stage"
##  $ grade   : Factor w/ 3 levels "I","II","III": 2 1 2 3 3 1 2 1 2 1 ...
##   ..- attr(*, "label")= chr "Grade"
##  $ response: int [1:200] 0 1 0 1 1 0 0 0 0 0 ...
##   ..- attr(*, "label")= chr "Tumor Response"
##  $ death   : int [1:200] 0 0 0 1 1 1 0 1 0 1 ...
##   ..- attr(*, "label")= chr "Patient Died"
##  $ ttdeath : num [1:200] 24 24 24 17.6 16.4 ...
##   ..- attr(*, "label")= chr "Months to Death/Censor"
\end{verbatim}

\begin{Shaded}
\begin{Highlighting}[]
\CommentTok{\#Recodage des variables et labellisation}

\NormalTok{trial }\OtherTok{\textless{}{-}}\NormalTok{ trial }\SpecialCharTok{\%\textgreater{}\%}
  \FunctionTok{mutate}\NormalTok{(}
    \AttributeTok{trt =} \FunctionTok{as.character}\NormalTok{(trt),}
    \AttributeTok{stage =} \FunctionTok{as.character}\NormalTok{(stage),}
    \AttributeTok{grade =} \FunctionTok{as.character}\NormalTok{(grade),}
    \AttributeTok{response=}\FunctionTok{as.character}\NormalTok{(response),}
    \AttributeTok{death=}\FunctionTok{as.character}\NormalTok{(death)}
\NormalTok{  ) }\SpecialCharTok{\%\textgreater{}\%}
  \FunctionTok{set\_variable\_labels}\NormalTok{(}
    \AttributeTok{trt =} \StringTok{"Traitement attribué"}\NormalTok{,}
    \AttributeTok{age =} \StringTok{"Age du patient"}\NormalTok{,}
    \AttributeTok{stage =} \StringTok{"Stade de la maladie"}\NormalTok{,}
    \AttributeTok{grade =} \StringTok{"Gravité de la maladie"}\NormalTok{,}
    \AttributeTok{response=}\StringTok{"Réponse de la tumeur"}\NormalTok{,}
    \AttributeTok{death=}\StringTok{"En vie ou mort"}\NormalTok{,}
    \AttributeTok{ttdeath=}\StringTok{"Total décès"}\NormalTok{,}
    \AttributeTok{marker=}\StringTok{"Marqueur sanguin"}
\NormalTok{  ) }\SpecialCharTok{\%\textgreater{}\%}
  \FunctionTok{set\_value\_labels}\NormalTok{(}
    \AttributeTok{trt =} \FunctionTok{c}\NormalTok{(}\StringTok{"A"} \OtherTok{=} \StringTok{"Drug A"}\NormalTok{, }\StringTok{"B"} \OtherTok{=} \StringTok{"Drug B"}\NormalTok{),}
    \AttributeTok{stage =} \FunctionTok{c}\NormalTok{(}\StringTok{"0"} \OtherTok{=} \StringTok{"T1"}\NormalTok{, }\StringTok{"1"} \OtherTok{=} \StringTok{"T2"}\NormalTok{, }\StringTok{"2"} \OtherTok{=} \StringTok{"T3"}\NormalTok{, }\StringTok{"3"} \OtherTok{=} \StringTok{"T4"}\NormalTok{),}
    \AttributeTok{grade =} \FunctionTok{c}\NormalTok{(}\StringTok{"Gravité1"} \OtherTok{=} \StringTok{"I"}\NormalTok{, }\StringTok{"Gravité2"} \OtherTok{=} \StringTok{"II"}\NormalTok{, }\StringTok{"Gravité3"} \OtherTok{=} \StringTok{"III"}\NormalTok{)}
\NormalTok{  ) }\SpecialCharTok{\%\textgreater{}\%}
  \FunctionTok{mutate}\NormalTok{(}
    \AttributeTok{trt =}\NormalTok{ labelled}\SpecialCharTok{::}\FunctionTok{to\_factor}\NormalTok{(trt),}
    \AttributeTok{stage =}\NormalTok{ labelled}\SpecialCharTok{::}\FunctionTok{to\_factor}\NormalTok{(stage),}
    \AttributeTok{grade =}\NormalTok{ labelled}\SpecialCharTok{::}\FunctionTok{to\_factor}\NormalTok{(grade)}
\NormalTok{  )}

\CommentTok{\# Forcer la mise à jour des étiquettes (facultatif)}
\FunctionTok{force}\NormalTok{(trial)}
\end{Highlighting}
\end{Shaded}

\begin{verbatim}
## # A tibble: 200 x 8
##    trt     age marker stage grade    response death ttdeath
##    <fct> <dbl>  <dbl> <fct> <fct>    <chr>    <chr>   <dbl>
##  1 A        23  0.16  0     Gravité2 0        0        24  
##  2 B         9  1.11  1     Gravité1 1        0        24  
##  3 A        31  0.277 0     Gravité2 0        0        24  
##  4 A        NA  2.07  2     Gravité3 1        1        17.6
##  5 A        51  2.77  3     Gravité3 1        1        16.4
##  6 B        39  0.613 3     Gravité1 0        1        15.6
##  7 A        37  0.354 0     Gravité2 0        0        24  
##  8 A        32  1.74  0     Gravité1 0        1        18.4
##  9 A        31  0.144 0     Gravité2 0        0        24  
## 10 B        34  0.205 2     Gravité1 0        1        10.5
## # i 190 more rows
\end{verbatim}

\begin{Shaded}
\begin{Highlighting}[]
\NormalTok{trial }\SpecialCharTok{\%\textgreater{}\%}
  \FunctionTok{tbl\_summary}\NormalTok{()}\SpecialCharTok{\%\textgreater{}\%}
  \FunctionTok{modify\_header}\NormalTok{(}\AttributeTok{label =} \StringTok{"**Variables**"}\NormalTok{)}
\end{Highlighting}
\end{Shaded}

\begin{table}[!t]
\fontsize{12.0pt}{14.4pt}\selectfont
\begin{tabular*}{\linewidth}{@{\extracolsep{\fill}}lc}
\toprule
\textbf{Variables} & \textbf{N = 200}\textsuperscript{\textit{1}} \\ 
\midrule\addlinespace[2.5pt]
Traitement attribué &  \\ 
    A & 98 (49\%) \\ 
    B & 102 (51\%) \\ 
Age du patient & 47 (38 – 57) \\ 
    Manquant & 11 \\ 
Marqueur sanguin & 0,64 (0,22 – 1,41) \\ 
    Manquant & 10 \\ 
Stade de la maladie &  \\ 
    0 & 53 (27\%) \\ 
    1 & 54 (27\%) \\ 
    2 & 43 (22\%) \\ 
    3 & 50 (25\%) \\ 
Gravité de la maladie &  \\ 
    Gravité1 & 68 (34\%) \\ 
    Gravité2 & 68 (34\%) \\ 
    Gravité3 & 64 (32\%) \\ 
Réponse de la tumeur &  \\ 
    0 & 132 (68\%) \\ 
    1 & 61 (32\%) \\ 
    Manquant & 7 \\ 
En vie ou mort &  \\ 
    0 & 88 (44\%) \\ 
    1 & 112 (56\%) \\ 
Total décès & 22,4 (15,9 – 24,0) \\ 
\bottomrule
\end{tabular*}
\begin{minipage}{\linewidth}
\textsuperscript{\textit{1}}n (\%); Médiane (Q1 -- Q3)\\
\end{minipage}
\end{table}

\begin{Shaded}
\begin{Highlighting}[]
\NormalTok{trial }\SpecialCharTok{\%\textgreater{}\%}
  \FunctionTok{tbl\_summary}\NormalTok{(}\AttributeTok{by=}\StringTok{"trt"}\NormalTok{)}\SpecialCharTok{\%\textgreater{}\%}
  \FunctionTok{modify\_header}\NormalTok{(}\AttributeTok{label =} \StringTok{"**Variables**"}\NormalTok{)}
\end{Highlighting}
\end{Shaded}

\begin{table}[!t]
\fontsize{12.0pt}{14.4pt}\selectfont
\begin{tabular*}{\linewidth}{@{\extracolsep{\fill}}lcc}
\toprule
\textbf{Variables} & \textbf{A}  N = 98\textsuperscript{\textit{1}} & \textbf{B}  N = 102\textsuperscript{\textit{1}} \\ 
\midrule\addlinespace[2.5pt]
Age du patient & 46 (37 – 60) & 48 (39 – 56) \\ 
    Manquant & 7 & 4 \\ 
Marqueur sanguin & 0,84 (0,23 – 1,60) & 0,52 (0,18 – 1,21) \\ 
    Manquant & 6 & 4 \\ 
Stade de la maladie &  &  \\ 
    0 & 28 (29\%) & 25 (25\%) \\ 
    1 & 25 (26\%) & 29 (28\%) \\ 
    2 & 22 (22\%) & 21 (21\%) \\ 
    3 & 23 (23\%) & 27 (26\%) \\ 
Gravité de la maladie &  &  \\ 
    Gravité1 & 35 (36\%) & 33 (32\%) \\ 
    Gravité2 & 32 (33\%) & 36 (35\%) \\ 
    Gravité3 & 31 (32\%) & 33 (32\%) \\ 
Réponse de la tumeur &  &  \\ 
    0 & 67 (71\%) & 65 (66\%) \\ 
    1 & 28 (29\%) & 33 (34\%) \\ 
    Manquant & 3 & 4 \\ 
En vie ou mort &  &  \\ 
    0 & 46 (47\%) & 42 (41\%) \\ 
    1 & 52 (53\%) & 60 (59\%) \\ 
Total décès & 23,5 (17,4 – 24,0) & 21,2 (14,5 – 24,0) \\ 
\bottomrule
\end{tabular*}
\begin{minipage}{\linewidth}
\textsuperscript{\textit{1}}Médiane (Q1 -- Q3); n (\%)\\
\end{minipage}
\end{table}

\begin{Shaded}
\begin{Highlighting}[]
\FunctionTok{library}\NormalTok{(GGally)}
\end{Highlighting}
\end{Shaded}

\begin{verbatim}
## Warning: le package 'GGally' a été compilé avec la version R 4.4.3
\end{verbatim}

\begin{verbatim}
## Registered S3 method overwritten by 'GGally':
##   method from   
##   +.gg   ggplot2
\end{verbatim}

\begin{Shaded}
\begin{Highlighting}[]
\NormalTok{trial}\SpecialCharTok{|\textgreater{}}\FunctionTok{ggbivariate}\NormalTok{(}\AttributeTok{outcome =} \StringTok{"trt"}\NormalTok{, }\AttributeTok{explanatory =} \FunctionTok{c}\NormalTok{(}\StringTok{"age"}\NormalTok{, }\StringTok{"marker"}\NormalTok{,}\StringTok{"grade"}\NormalTok{, }\StringTok{"stage"}\NormalTok{,}\StringTok{"ttdeath"}\NormalTok{,}\StringTok{"death"}\NormalTok{))}
\end{Highlighting}
\end{Shaded}

\begin{verbatim}
## Warning: Removed 11 rows containing non-finite outside the scale range
## (`stat_boxplot()`).
\end{verbatim}

\begin{verbatim}
## Warning: Removed 11 rows containing non-finite outside the scale range
## (`stat_boxplot()`).
\end{verbatim}

\begin{verbatim}
## Warning: Removed 10 rows containing non-finite outside the scale range
## (`stat_boxplot()`).
\end{verbatim}

\includegraphics{formR_files/figure-latex/unnamed-chunk-1-1.pdf}

\begin{Shaded}
\begin{Highlighting}[]
\NormalTok{trial}\SpecialCharTok{|\textgreater{}}\FunctionTok{ggtable}\NormalTok{(}\AttributeTok{columnsY =} \FunctionTok{c}\NormalTok{(}\StringTok{"age"}\NormalTok{, }\StringTok{"marker"}\NormalTok{,}\StringTok{"grade"}\NormalTok{, }\StringTok{"stage"}\NormalTok{,}\StringTok{"ttdeath"}\NormalTok{,}\StringTok{"death"}\NormalTok{), }\AttributeTok{columnsX =} \StringTok{"trt"}\NormalTok{,}
               \AttributeTok{fill=}\StringTok{"std.resid"}\NormalTok{,}
               \AttributeTok{cells=}\StringTok{"row.prop"}\NormalTok{,}
\NormalTok{               )}
\end{Highlighting}
\end{Shaded}

\includegraphics{formR_files/figure-latex/unnamed-chunk-1-2.pdf}

\begin{Shaded}
\begin{Highlighting}[]
\CommentTok{\#library(tidyverse)}
\CommentTok{\#library(labelled)}

\CommentTok{\#data("trial")}

\CommentTok{\#trial\_labeled \textless{}{-} trial \%\textgreater{}\%}
\CommentTok{\#  set\_variable\_labels(}
\CommentTok{\#    trt = "Traitement attribué",}
\CommentTok{\#    age = "Age du patient",}
\CommentTok{\#    marker = "Marqueur sanguin",}
\CommentTok{\#    grade = "Gravité de la maladie",}
\CommentTok{\#    stage = "Stade de la maladie",}
\CommentTok{\#    ttdeath = "Temps jusqu\textquotesingle{}au décès",}
\CommentTok{\#    death = "Décès"}
\CommentTok{\#  )}

\NormalTok{apply\_labels\_ggbivariate }\OtherTok{\textless{}{-}} \ControlFlowTok{function}\NormalTok{(p, data, outcome, explanatory) \{}
\NormalTok{  outcome\_label }\OtherTok{\textless{}{-}} \FunctionTok{var\_label}\NormalTok{(data[[outcome]])}
  \ControlFlowTok{for}\NormalTok{ (var }\ControlFlowTok{in}\NormalTok{ explanatory) \{}
\NormalTok{    var\_label\_text }\OtherTok{\textless{}{-}} \FunctionTok{var\_label}\NormalTok{(data[[var]])}
    \ControlFlowTok{if}\NormalTok{ (}\SpecialCharTok{!}\FunctionTok{is.null}\NormalTok{(var\_label\_text)) \{}
\NormalTok{      p }\OtherTok{\textless{}{-}}\NormalTok{ p }\SpecialCharTok{+} \FunctionTok{labs}\NormalTok{(}\AttributeTok{x =}\NormalTok{ var\_label\_text, }\AttributeTok{y =}\NormalTok{ outcome\_label)}
\NormalTok{    \}}
\NormalTok{  \}}
  \FunctionTok{return}\NormalTok{(p)}
\NormalTok{\}}

\NormalTok{plot }\OtherTok{\textless{}{-}}\NormalTok{ trial }\SpecialCharTok{\%\textgreater{}\%}
  \FunctionTok{ggbivariate}\NormalTok{(}
    \AttributeTok{outcome =} \StringTok{"trt"}\NormalTok{,}
    \AttributeTok{explanatory =} \FunctionTok{c}\NormalTok{(}\StringTok{"age"}\NormalTok{, }\StringTok{"marker"}\NormalTok{, }\StringTok{"grade"}\NormalTok{, }\StringTok{"stage"}\NormalTok{, }\StringTok{"ttdeath"}\NormalTok{, }\StringTok{"death"}\NormalTok{)}
\NormalTok{  )}

\NormalTok{plot }\OtherTok{\textless{}{-}} \FunctionTok{apply\_labels\_ggbivariate}\NormalTok{(plot, trial, }\StringTok{"trt"}\NormalTok{, }\FunctionTok{c}\NormalTok{(}\StringTok{"age"}\NormalTok{, }\StringTok{"marker"}\NormalTok{, }\StringTok{"grade"}\NormalTok{, }\StringTok{"stage"}\NormalTok{, }\StringTok{"ttdeath"}\NormalTok{, }\StringTok{"death"}\NormalTok{))}

\FunctionTok{print}\NormalTok{(plot)}
\end{Highlighting}
\end{Shaded}

\begin{verbatim}
## Warning: Removed 11 rows containing non-finite outside the scale range
## (`stat_boxplot()`).
\end{verbatim}

\begin{verbatim}
## Warning: Removed 11 rows containing non-finite outside the scale range
## (`stat_boxplot()`).
\end{verbatim}

\begin{verbatim}
## Warning: Removed 10 rows containing non-finite outside the scale range
## (`stat_boxplot()`).
\end{verbatim}

\includegraphics{formR_files/figure-latex/unnamed-chunk-1-3.pdf}

\begin{Shaded}
\begin{Highlighting}[]
\FunctionTok{library}\NormalTok{(ggplot2)}

\FunctionTok{ggplot}\NormalTok{(trial) }\SpecialCharTok{+}
  \FunctionTok{aes}\NormalTok{(}\AttributeTok{x =}\NormalTok{ age, }\AttributeTok{y =}\NormalTok{ trt, }\AttributeTok{fill =}\NormalTok{ grade) }\SpecialCharTok{+}
  \FunctionTok{geom\_boxplot}\NormalTok{() }\SpecialCharTok{+}
  \FunctionTok{scale\_fill\_manual}\NormalTok{(}\AttributeTok{values =} \FunctionTok{c}\NormalTok{(}\AttributeTok{I =} \StringTok{"\#F8766D"}\NormalTok{, }
                               \AttributeTok{II =} \StringTok{"\#00C19F"}\NormalTok{, }\AttributeTok{III =} \StringTok{"\#FF61C3"}\NormalTok{)) }\SpecialCharTok{+}
  \FunctionTok{labs}\NormalTok{(}\AttributeTok{x =} \StringTok{"Age"}\NormalTok{, }\AttributeTok{y =} \StringTok{"Traitement"}\NormalTok{, }\AttributeTok{title =} \StringTok{"Répartition des traitements (trt) en fonction de l\textquotesingle{}âge, selon la gravité de la maladie (grade)"}\NormalTok{, }
       \AttributeTok{caption =} \StringTok{"FormationR"}\NormalTok{, }\AttributeTok{fill =} \StringTok{"Gravité de la tumeur"}\NormalTok{) }\SpecialCharTok{+}
  \FunctionTok{theme\_linedraw}\NormalTok{() }\SpecialCharTok{+}
  \FunctionTok{theme}\NormalTok{(}\AttributeTok{axis.text.y =} \FunctionTok{element\_text}\NormalTok{(}\AttributeTok{face =} \StringTok{"bold"}\NormalTok{), }
        \AttributeTok{axis.text.x =} \FunctionTok{element\_text}\NormalTok{(}\AttributeTok{face =} \StringTok{"bold"}\NormalTok{, }\AttributeTok{size =} \DecValTok{15}\DataTypeTok{L}\NormalTok{), }\AttributeTok{legend.text =} \FunctionTok{element\_text}\NormalTok{(}\AttributeTok{face =} \StringTok{"bold"}\NormalTok{), }
        \AttributeTok{legend.title =} \FunctionTok{element\_text}\NormalTok{(}\AttributeTok{face =} \StringTok{"bold"}\NormalTok{))}
\end{Highlighting}
\end{Shaded}

\begin{verbatim}
## Warning: Removed 11 rows containing non-finite outside the scale range
## (`stat_boxplot()`).
\end{verbatim}

\begin{verbatim}
## Warning: No shared levels found between `names(values)` of the manual scale and the
## data's fill values.
## No shared levels found between `names(values)` of the manual scale and the
## data's fill values.
\end{verbatim}

\includegraphics{formR_files/figure-latex/unnamed-chunk-1-4.pdf}

\begin{Shaded}
\begin{Highlighting}[]
\CommentTok{\#Modèle de regression linéaire simple et multiple.}


\CommentTok{\# Préparation des données : sélection des variables et suppression des valeurs manquantes}
\NormalTok{trial\_regression }\OtherTok{\textless{}{-}}\NormalTok{ trial }\SpecialCharTok{\%\textgreater{}\%}
  \FunctionTok{select}\NormalTok{(age, marker,ttdeath) }\SpecialCharTok{\%\textgreater{}\%}
  \FunctionTok{drop\_na}\NormalTok{()}

\CommentTok{\# Création du modèle}
\NormalTok{modele\_regression }\OtherTok{\textless{}{-}} \FunctionTok{lm}\NormalTok{(ttdeath }\SpecialCharTok{\textasciitilde{}}\NormalTok{ marker}\SpecialCharTok{+}\NormalTok{age, }\AttributeTok{data =}\NormalTok{ trial\_regression)}

\CommentTok{\# Affichage du résumé du modèle}
\FunctionTok{summary}\NormalTok{(modele\_regression)}
\end{Highlighting}
\end{Shaded}

\begin{verbatim}
## 
## Call:
## lm(formula = ttdeath ~ marker + age, data = trial_regression)
## 
## Residuals:
##     Min      1Q  Median      3Q     Max 
## -15.937  -3.534   3.144   4.007   4.594 
## 
## Coefficients:
##             Estimate Std. Error t value Pr(>|t|)    
## (Intercept) 20.64546    1.38933  14.860   <2e-16 ***
## marker       0.24828    0.45000   0.552    0.582    
## age         -0.01925    0.02694  -0.715    0.476    
## ---
## Signif. codes:  0 '***' 0.001 '**' 0.01 '*' 0.05 '.' 0.1 ' ' 1
## 
## Residual standard error: 5.199 on 176 degrees of freedom
## Multiple R-squared:  0.004624,   Adjusted R-squared:  -0.006687 
## F-statistic: 0.4088 on 2 and 176 DF,  p-value: 0.6651
\end{verbatim}

\begin{Shaded}
\begin{Highlighting}[]
\FunctionTok{library}\NormalTok{(tidyverse)}
\FunctionTok{library}\NormalTok{(gtsummary)}

\FunctionTok{data}\NormalTok{(}\StringTok{"trial"}\NormalTok{)}

\CommentTok{\# Conversion des variables catégorielles en facteurs}
\NormalTok{trial}\SpecialCharTok{$}\NormalTok{stage }\OtherTok{\textless{}{-}} \FunctionTok{as.factor}\NormalTok{(trial}\SpecialCharTok{$}\NormalTok{stage)}
\NormalTok{trial}\SpecialCharTok{$}\NormalTok{grade }\OtherTok{\textless{}{-}} \FunctionTok{as.factor}\NormalTok{(trial}\SpecialCharTok{$}\NormalTok{grade)}
\NormalTok{trial}\SpecialCharTok{$}\NormalTok{trt }\OtherTok{\textless{}{-}} \FunctionTok{as.factor}\NormalTok{(trial}\SpecialCharTok{$}\NormalTok{trt)}
\NormalTok{trial}\SpecialCharTok{$}\NormalTok{response }\OtherTok{\textless{}{-}} \FunctionTok{as.factor}\NormalTok{(trial}\SpecialCharTok{$}\NormalTok{response)}

\CommentTok{\# Gestion des valeurs manquantes}
\NormalTok{trial }\OtherTok{\textless{}{-}}\NormalTok{ trial }\SpecialCharTok{\%\textgreater{}\%} \FunctionTok{drop\_na}\NormalTok{()}

\CommentTok{\# Vérification des types de variables}
\FunctionTok{str}\NormalTok{(trial)}
\end{Highlighting}
\end{Shaded}

\begin{verbatim}
## tibble [173 x 8] (S3: tbl_df/tbl/data.frame)
##  $ trt     : Factor w/ 2 levels "Drug A","Drug B": 1 2 1 1 2 1 1 1 2 2 ...
##  $ age     : num [1:173] 23 9 31 51 39 37 32 31 34 42 ...
##   ..- attr(*, "label")= chr "Age"
##  $ marker  : num [1:173] 0.16 1.107 0.277 2.767 0.613 ...
##   ..- attr(*, "label")= chr "Marker Level (ng/mL)"
##  $ stage   : Factor w/ 4 levels "T1","T2","T3",..: 1 2 1 4 4 1 1 1 3 1 ...
##   ..- attr(*, "label")= chr "T Stage"
##  $ grade   : Factor w/ 3 levels "I","II","III": 2 1 2 3 1 2 1 2 1 3 ...
##   ..- attr(*, "label")= chr "Grade"
##  $ response: Factor w/ 2 levels "0","1": 1 2 1 2 1 1 1 1 1 1 ...
##  $ death   : int [1:173] 0 0 0 1 1 0 1 0 1 0 ...
##   ..- attr(*, "label")= chr "Patient Died"
##  $ ttdeath : num [1:173] 24 24 24 16.4 15.6 ...
##   ..- attr(*, "label")= chr "Months to Death/Censor"
\end{verbatim}

\begin{Shaded}
\begin{Highlighting}[]
\CommentTok{\# Modèle de régression logistique}
\NormalTok{modele\_response }\OtherTok{\textless{}{-}} \FunctionTok{glm}\NormalTok{(response }\SpecialCharTok{\textasciitilde{}}\NormalTok{ age }\SpecialCharTok{+}\NormalTok{ marker }\SpecialCharTok{+}\NormalTok{ stage }\SpecialCharTok{+}\NormalTok{ grade }\SpecialCharTok{+}\NormalTok{ trt, }\AttributeTok{data =}\NormalTok{ trial, }\AttributeTok{family =} \StringTok{"binomial"}\NormalTok{)}

\CommentTok{\# Affichage du résumé du modèle}
\FunctionTok{summary}\NormalTok{(modele\_response)}
\end{Highlighting}
\end{Shaded}

\begin{verbatim}
## 
## Call:
## glm(formula = response ~ age + marker + stage + grade + trt, 
##     family = "binomial", data = trial)
## 
## Coefficients:
##             Estimate Std. Error z value Pr(>|z|)  
## (Intercept) -1.85393    0.73976  -2.506   0.0122 *
## age          0.01901    0.01193   1.593   0.1112  
## marker       0.34596    0.19688   1.757   0.0789 .
## stageT2     -0.80936    0.47127  -1.717   0.0859 .
## stageT3     -0.14589    0.48377  -0.302   0.7630  
## stageT4     -0.43985    0.47316  -0.930   0.3526  
## gradeII      0.03926    0.42852   0.092   0.9270  
## gradeIII     0.03984    0.40810   0.098   0.9222  
## trtDrug B    0.29470    0.34084   0.865   0.3872  
## ---
## Signif. codes:  0 '***' 0.001 '**' 0.01 '*' 0.05 '.' 0.1 ' ' 1
## 
## (Dispersion parameter for binomial family taken to be 1)
## 
##     Null deviance: 214.80  on 172  degrees of freedom
## Residual deviance: 205.89  on 164  degrees of freedom
## AIC: 223.89
## 
## Number of Fisher Scoring iterations: 4
\end{verbatim}

\begin{Shaded}
\begin{Highlighting}[]
\FunctionTok{installed.packages}\NormalTok{(}\StringTok{"broom.helpers"}\NormalTok{)}
\end{Highlighting}
\end{Shaded}

\begin{verbatim}
##      Package LibPath Version Priority Depends Imports LinkingTo Suggests
##      Enhances License License_is_FOSS License_restricts_use OS_type Archs
##      MD5sum NeedsCompilation Built
\end{verbatim}

\begin{Shaded}
\begin{Highlighting}[]
\FunctionTok{library}\NormalTok{(broom.helpers)}
\end{Highlighting}
\end{Shaded}

\begin{verbatim}
## Warning: le package 'broom.helpers' a été compilé avec la version R 4.4.3
\end{verbatim}

\begin{verbatim}
## 
## Attachement du package : 'broom.helpers'
## 
## Les objets suivants sont masqués depuis 'package:gtsummary':
## 
##     all_categorical, all_continuous, all_contrasts, all_dichotomous,
##     all_interaction, all_intercepts
\end{verbatim}

\begin{Shaded}
\begin{Highlighting}[]
\NormalTok{modele\_response}\SpecialCharTok{\%\textgreater{}\%}
  \FunctionTok{tbl\_regression}\NormalTok{(}
    \AttributeTok{exponentiate=}\ConstantTok{TRUE}\NormalTok{, }\AttributeTok{add\_estimate\_to\_reference\_rows =} \ConstantTok{TRUE}
\NormalTok{  )}\SpecialCharTok{\%\textgreater{}\%}
  \FunctionTok{add\_global\_p}\NormalTok{()}
\end{Highlighting}
\end{Shaded}

\begin{table}[!t]
\fontsize{12.0pt}{14.4pt}\selectfont
\begin{tabular*}{\linewidth}{@{\extracolsep{\fill}}lccc}
\toprule
\textbf{Caractéristique} & \textbf{OR} & \textbf{95\% IC} & \textbf{p-valeur} \\ 
\midrule\addlinespace[2.5pt]
Age & 1,02 & 1,00 – 1,04 & 0,11 \\ 
Marker Level (ng/mL) & 1,41 & 0,96 – 2,09 & 0,080 \\ 
T Stage &  &  & 0,3 \\ 
    T1 & 1,00 & — &  \\ 
    T2 & 0,45 & 0,17 – 1,11 &  \\ 
    T3 & 0,86 & 0,33 – 2,22 &  \\ 
    T4 & 0,64 & 0,25 – 1,62 &  \\ 
Grade &  &  & >0,9 \\ 
    I & 1,00 & — &  \\ 
    II & 1,04 & 0,45 – 2,42 &  \\ 
    III & 1,04 & 0,47 – 2,32 &  \\ 
trt &  &  & 0,4 \\ 
    Drug A & 1,00 & — &  \\ 
    Drug B & 1,34 & 0,69 – 2,64 &  \\ 
\bottomrule
\end{tabular*}
\begin{minipage}{\linewidth}
Abréviations: IC = intervalle de confiance, OR = rapport de cotes\\
\end{minipage}
\end{table}

\begin{Shaded}
\begin{Highlighting}[]
\FunctionTok{library}\NormalTok{(tidyverse)}
\FunctionTok{library}\NormalTok{(gtsummary)}

\FunctionTok{data}\NormalTok{(}\StringTok{"trial"}\NormalTok{)}

\CommentTok{\# Préparation des données : sélection des variables et conversion en facteurs}
\NormalTok{trial\_anova }\OtherTok{\textless{}{-}}\NormalTok{ trial }\SpecialCharTok{\%\textgreater{}\%}
  \FunctionTok{select}\NormalTok{(age, trt, grade) }\SpecialCharTok{\%\textgreater{}\%}
  \FunctionTok{mutate}\NormalTok{(}
    \AttributeTok{trt =} \FunctionTok{as.factor}\NormalTok{(trt),}
    \AttributeTok{grade =} \FunctionTok{as.factor}\NormalTok{(grade)}
\NormalTok{  ) }\SpecialCharTok{\%\textgreater{}\%}
  \FunctionTok{drop\_na}\NormalTok{()}


\CommentTok{\# ANOVA à un facteur : age \textasciitilde{} trt}
\NormalTok{modele\_anova\_1 }\OtherTok{\textless{}{-}} \FunctionTok{aov}\NormalTok{(age }\SpecialCharTok{\textasciitilde{}}\NormalTok{ trt, }\AttributeTok{data =}\NormalTok{ trial\_anova)}

\CommentTok{\# Affichage du résumé du modèle}
\FunctionTok{summary}\NormalTok{(modele\_anova\_1)}
\end{Highlighting}
\end{Shaded}

\begin{verbatim}
##              Df Sum Sq Mean Sq F value Pr(>F)
## trt           1      9    9.05   0.044  0.834
## Residuals   187  38499  205.88
\end{verbatim}

\begin{Shaded}
\begin{Highlighting}[]
\CommentTok{\# ANOVA à deux facteurs : age \textasciitilde{} trt + grade}
\NormalTok{modele\_anova\_2 }\OtherTok{\textless{}{-}} \FunctionTok{aov}\NormalTok{(age }\SpecialCharTok{\textasciitilde{}}\NormalTok{ trt }\SpecialCharTok{+}\NormalTok{ grade, }\AttributeTok{data =}\NormalTok{ trial\_anova)}

\CommentTok{\# Affichage du résumé du modèle}
\FunctionTok{summary}\NormalTok{(modele\_anova\_2)}
\end{Highlighting}
\end{Shaded}

\begin{verbatim}
##              Df Sum Sq Mean Sq F value Pr(>F)
## trt           1      9    9.05   0.044  0.835
## grade         2    128   64.07   0.309  0.735
## Residuals   185  38371  207.41
\end{verbatim}

\begin{Shaded}
\begin{Highlighting}[]
\CommentTok{\# ANOVA à deux facteurs avec interaction : age \textasciitilde{} trt * grade}
\NormalTok{modele\_anova\_inter }\OtherTok{\textless{}{-}} \FunctionTok{aov}\NormalTok{(age }\SpecialCharTok{\textasciitilde{}}\NormalTok{ trt }\SpecialCharTok{*}\NormalTok{ grade, }\AttributeTok{data =}\NormalTok{ trial\_anova)}

\CommentTok{\# Affichage du résumé du modèle}
\FunctionTok{summary}\NormalTok{(modele\_anova\_inter)}
\end{Highlighting}
\end{Shaded}

\begin{verbatim}
##              Df Sum Sq Mean Sq F value Pr(>F)
## trt           1      9     9.1   0.044  0.834
## grade         2    128    64.1   0.313  0.732
## trt:grade     2    929   464.7   2.271  0.106
## Residuals   183  37442   204.6
\end{verbatim}

\begin{Shaded}
\begin{Highlighting}[]
\CommentTok{\# Graphiques de diagnostic}
\FunctionTok{plot}\NormalTok{(modele\_anova\_1)}
\end{Highlighting}
\end{Shaded}

\includegraphics{formR_files/figure-latex/unnamed-chunk-1-5.pdf}
\includegraphics{formR_files/figure-latex/unnamed-chunk-1-6.pdf}
\includegraphics{formR_files/figure-latex/unnamed-chunk-1-7.pdf}
\includegraphics{formR_files/figure-latex/unnamed-chunk-1-8.pdf}

\begin{Shaded}
\begin{Highlighting}[]
\FunctionTok{plot}\NormalTok{(modele\_anova\_2)}
\end{Highlighting}
\end{Shaded}

\includegraphics{formR_files/figure-latex/unnamed-chunk-1-9.pdf}
\includegraphics{formR_files/figure-latex/unnamed-chunk-1-10.pdf}
\includegraphics{formR_files/figure-latex/unnamed-chunk-1-11.pdf}
\includegraphics{formR_files/figure-latex/unnamed-chunk-1-12.pdf}

\begin{Shaded}
\begin{Highlighting}[]
\FunctionTok{plot}\NormalTok{(modele\_anova\_inter)}
\end{Highlighting}
\end{Shaded}

\includegraphics{formR_files/figure-latex/unnamed-chunk-1-13.pdf}
\includegraphics{formR_files/figure-latex/unnamed-chunk-1-14.pdf}
\includegraphics{formR_files/figure-latex/unnamed-chunk-1-15.pdf}
\includegraphics{formR_files/figure-latex/unnamed-chunk-1-16.pdf}

\begin{Shaded}
\begin{Highlighting}[]
\CommentTok{\# Test de Levene pour l\textquotesingle{}homogénéité des variances}
\FunctionTok{library}\NormalTok{(car)}
\FunctionTok{leveneTest}\NormalTok{(age }\SpecialCharTok{\textasciitilde{}}\NormalTok{ trt, }\AttributeTok{data =}\NormalTok{ trial\_anova)}
\end{Highlighting}
\end{Shaded}

\begin{verbatim}
## Levene's Test for Homogeneity of Variance (center = median)
##        Df F value Pr(>F)
## group   1  0.7053 0.4021
##       187
\end{verbatim}

\begin{Shaded}
\begin{Highlighting}[]
\FunctionTok{leveneTest}\NormalTok{(age }\SpecialCharTok{\textasciitilde{}}\NormalTok{ trt }\SpecialCharTok{*}\NormalTok{ grade, }\AttributeTok{data =}\NormalTok{ trial\_anova)}
\end{Highlighting}
\end{Shaded}

\begin{verbatim}
## Levene's Test for Homogeneity of Variance (center = median)
##        Df F value Pr(>F)
## group   5  0.4313 0.8264
##       183
\end{verbatim}

\begin{Shaded}
\begin{Highlighting}[]
\FunctionTok{library}\NormalTok{(tidyverse)}
\FunctionTok{library}\NormalTok{(survival)}
\FunctionTok{library}\NormalTok{(survminer)}
\end{Highlighting}
\end{Shaded}

\begin{verbatim}
## Warning: le package 'survminer' a été compilé avec la version R 4.4.2
\end{verbatim}

\begin{verbatim}
## Le chargement a nécessité le package : ggpubr
\end{verbatim}

\begin{verbatim}
## Warning: le package 'ggpubr' a été compilé avec la version R 4.4.3
\end{verbatim}

\begin{verbatim}
## 
## Attachement du package : 'survminer'
## 
## L'objet suivant est masqué depuis 'package:survival':
## 
##     myeloma
\end{verbatim}

\begin{Shaded}
\begin{Highlighting}[]
\FunctionTok{library}\NormalTok{(gtsummary)}

\FunctionTok{data}\NormalTok{(}\StringTok{"trial"}\NormalTok{)}

\CommentTok{\# Préparation des données : sélection des variables et conversion en facteurs}
\NormalTok{trial\_survie }\OtherTok{\textless{}{-}}\NormalTok{ trial }\SpecialCharTok{\%\textgreater{}\%}
  \FunctionTok{select}\NormalTok{(ttdeath, death, trt) }\SpecialCharTok{\%\textgreater{}\%}
  \FunctionTok{mutate}\NormalTok{(}
    \AttributeTok{trt =} \FunctionTok{as.factor}\NormalTok{(trt)}
\NormalTok{  ) }\SpecialCharTok{\%\textgreater{}\%}
  \FunctionTok{drop\_na}\NormalTok{()}


\CommentTok{\# Création de l\textquotesingle{}objet de survie}
\NormalTok{objet\_survie }\OtherTok{\textless{}{-}} \FunctionTok{Surv}\NormalTok{(}\AttributeTok{time =}\NormalTok{ trial\_survie}\SpecialCharTok{$}\NormalTok{ttdeath, }\AttributeTok{event =}\NormalTok{ trial\_survie}\SpecialCharTok{$}\NormalTok{death)}


\CommentTok{\# Estimation des courbes de survie}
\NormalTok{modele\_survie }\OtherTok{\textless{}{-}} \FunctionTok{survfit}\NormalTok{(objet\_survie }\SpecialCharTok{\textasciitilde{}}\NormalTok{ trt, }\AttributeTok{data =}\NormalTok{ trial\_survie)}


\CommentTok{\# Visualisation des courbes de survie}
\FunctionTok{ggsurvplot}\NormalTok{(modele\_survie, }\AttributeTok{data =}\NormalTok{ trial\_survie,}
           \AttributeTok{pval =} \ConstantTok{TRUE}\NormalTok{, }\AttributeTok{conf.int =} \ConstantTok{TRUE}\NormalTok{,}
           \AttributeTok{risk.table =} \ConstantTok{TRUE}\NormalTok{,}
           \AttributeTok{ggtheme =} \FunctionTok{theme\_bw}\NormalTok{())}
\end{Highlighting}
\end{Shaded}

\includegraphics{formR_files/figure-latex/unnamed-chunk-1-17.pdf}

\begin{Shaded}
\begin{Highlighting}[]
\CommentTok{\# Test du log{-}rank}
\NormalTok{test\_logrank }\OtherTok{\textless{}{-}} \FunctionTok{survdiff}\NormalTok{(objet\_survie }\SpecialCharTok{\textasciitilde{}}\NormalTok{ trt, }\AttributeTok{data =}\NormalTok{ trial\_survie)}
\NormalTok{test\_logrank}
\end{Highlighting}
\end{Shaded}

\begin{verbatim}
## Call:
## survdiff(formula = objet_survie ~ trt, data = trial_survie)
## 
##              N Observed Expected (O-E)^2/E (O-E)^2/V
## trt=Drug A  98       52     58.2     0.665      1.39
## trt=Drug B 102       60     53.8     0.720      1.39
## 
##  Chisq= 1.4  on 1 degrees of freedom, p= 0.2
\end{verbatim}

\begin{Shaded}
\begin{Highlighting}[]
\FunctionTok{library}\NormalTok{(tidyverse)}
\FunctionTok{library}\NormalTok{(caret)}
\end{Highlighting}
\end{Shaded}

\begin{verbatim}
## Warning: le package 'caret' a été compilé avec la version R 4.4.3
\end{verbatim}

\begin{verbatim}
## Le chargement a nécessité le package : lattice
## 
## Attachement du package : 'caret'
## 
## L'objet suivant est masqué depuis 'package:survival':
## 
##     cluster
## 
## L'objet suivant est masqué depuis 'package:purrr':
## 
##     lift
\end{verbatim}

\begin{Shaded}
\begin{Highlighting}[]
\FunctionTok{library}\NormalTok{(gtsummary)}
\FunctionTok{library}\NormalTok{(ggplot2)}
\FunctionTok{library}\NormalTok{(pROC)}
\end{Highlighting}
\end{Shaded}

\begin{verbatim}
## Warning: le package 'pROC' a été compilé avec la version R 4.4.3
\end{verbatim}

\begin{verbatim}
## Type 'citation("pROC")' for a citation.
## 
## Attachement du package : 'pROC'
## 
## Les objets suivants sont masqués depuis 'package:stats':
## 
##     cov, smooth, var
\end{verbatim}

\begin{Shaded}
\begin{Highlighting}[]
\FunctionTok{data}\NormalTok{(}\StringTok{"trial"}\NormalTok{)}

\CommentTok{\# Préparation des données (comme précédemment)}
\NormalTok{trial\_ml }\OtherTok{\textless{}{-}}\NormalTok{ trial }\SpecialCharTok{\%\textgreater{}\%}
  \FunctionTok{select}\NormalTok{(age, marker, stage, grade, trt, response) }\SpecialCharTok{\%\textgreater{}\%}
  \FunctionTok{mutate\_if}\NormalTok{(is.character, as.factor) }\SpecialCharTok{\%\textgreater{}\%}
  \FunctionTok{drop\_na}\NormalTok{()}

\FunctionTok{set.seed}\NormalTok{(}\DecValTok{123}\NormalTok{)}
\NormalTok{trainIndex }\OtherTok{\textless{}{-}} \FunctionTok{createDataPartition}\NormalTok{(trial\_ml}\SpecialCharTok{$}\NormalTok{response, }\AttributeTok{p =} \FloatTok{0.7}\NormalTok{, }\AttributeTok{list =} \ConstantTok{FALSE}\NormalTok{)}
\NormalTok{train\_data }\OtherTok{\textless{}{-}}\NormalTok{ trial\_ml[trainIndex, ]}
\NormalTok{test\_data }\OtherTok{\textless{}{-}}\NormalTok{ trial\_ml[}\SpecialCharTok{{-}}\NormalTok{trainIndex, ]}

\CommentTok{\# Modèle de régression logistique}
\NormalTok{modele\_logistique }\OtherTok{\textless{}{-}} \FunctionTok{glm}\NormalTok{(response }\SpecialCharTok{\textasciitilde{}}\NormalTok{ ., }\AttributeTok{data =}\NormalTok{ train\_data, }\AttributeTok{family =} \StringTok{"binomial"}\NormalTok{)}

\CommentTok{\# Prédictions sur l\textquotesingle{}ensemble de test}
\NormalTok{predictions\_logistique\_prob }\OtherTok{\textless{}{-}} \FunctionTok{predict}\NormalTok{(modele\_logistique, }\AttributeTok{newdata =}\NormalTok{ test\_data, }\AttributeTok{type =} \StringTok{"response"}\NormalTok{)}
\NormalTok{predictions\_logistique }\OtherTok{\textless{}{-}} \FunctionTok{ifelse}\NormalTok{(predictions\_logistique\_prob }\SpecialCharTok{\textgreater{}} \FloatTok{0.5}\NormalTok{, }\StringTok{"1"}\NormalTok{, }\StringTok{"0"}\NormalTok{)}

\CommentTok{\# Convertir les prédictions et les valeurs réelles en facteurs}
\NormalTok{predictions\_logistique }\OtherTok{\textless{}{-}} \FunctionTok{as.factor}\NormalTok{(predictions\_logistique)}
\NormalTok{test\_data}\SpecialCharTok{$}\NormalTok{response }\OtherTok{\textless{}{-}} \FunctionTok{as.factor}\NormalTok{(test\_data}\SpecialCharTok{$}\NormalTok{response)}

\CommentTok{\# Trouver et appliquer les mêmes niveaux}
\NormalTok{unique\_levels }\OtherTok{\textless{}{-}} \FunctionTok{unique}\NormalTok{(}\FunctionTok{c}\NormalTok{(}\FunctionTok{levels}\NormalTok{(predictions\_logistique), }\FunctionTok{levels}\NormalTok{(test\_data}\SpecialCharTok{$}\NormalTok{response)))}

\NormalTok{predictions\_logistique }\OtherTok{\textless{}{-}} \FunctionTok{factor}\NormalTok{(predictions\_logistique, }\AttributeTok{levels =}\NormalTok{ unique\_levels)}
\NormalTok{test\_data}\SpecialCharTok{$}\NormalTok{response }\OtherTok{\textless{}{-}} \FunctionTok{factor}\NormalTok{(test\_data}\SpecialCharTok{$}\NormalTok{response, }\AttributeTok{levels =}\NormalTok{ unique\_levels)}

\CommentTok{\# Matrice de confusion}
\NormalTok{cm }\OtherTok{\textless{}{-}} \FunctionTok{confusionMatrix}\NormalTok{(predictions\_logistique, test\_data}\SpecialCharTok{$}\NormalTok{response)}

\CommentTok{\# Visualisation de la matrice de confusion (exemple avec ggplot2)}
\NormalTok{cm\_data }\OtherTok{\textless{}{-}} \FunctionTok{data.frame}\NormalTok{(}
  \AttributeTok{Prediction =}\NormalTok{ predictions\_logistique,}
  \AttributeTok{Reference =}\NormalTok{ test\_data}\SpecialCharTok{$}\NormalTok{response}
\NormalTok{)}

\FunctionTok{ggplot}\NormalTok{(cm\_data, }\FunctionTok{aes}\NormalTok{(}\AttributeTok{x =}\NormalTok{ Reference, }\AttributeTok{fill =}\NormalTok{ Prediction)) }\SpecialCharTok{+}
  \FunctionTok{geom\_bar}\NormalTok{(}\AttributeTok{position =} \StringTok{"dodge"}\NormalTok{) }\SpecialCharTok{+}
  \FunctionTok{labs}\NormalTok{(}\AttributeTok{title =} \StringTok{"Matrice de confusion"}\NormalTok{, }\AttributeTok{x =} \StringTok{"Valeurs réelles"}\NormalTok{, }\AttributeTok{y =} \StringTok{"Nombre de prédictions"}\NormalTok{)}
\end{Highlighting}
\end{Shaded}

\includegraphics{formR_files/figure-latex/unnamed-chunk-1-18.pdf}

\begin{Shaded}
\begin{Highlighting}[]
\CommentTok{\# Courbe ROC}
\NormalTok{roc\_obj }\OtherTok{\textless{}{-}} \FunctionTok{roc}\NormalTok{(test\_data}\SpecialCharTok{$}\NormalTok{response, predictions\_logistique\_prob)}
\end{Highlighting}
\end{Shaded}

\begin{verbatim}
## Setting levels: control = 0, case = 1
## Setting direction: controls < cases
\end{verbatim}

\begin{Shaded}
\begin{Highlighting}[]
\FunctionTok{plot}\NormalTok{(roc\_obj, }\AttributeTok{main =} \StringTok{"Courbe ROC"}\NormalTok{)}
\end{Highlighting}
\end{Shaded}

\includegraphics{formR_files/figure-latex/unnamed-chunk-1-19.pdf}

\begin{Shaded}
\begin{Highlighting}[]
\CommentTok{\# Importance des variables (exemple avec les forêts aléatoires)}
\NormalTok{modele\_foret }\OtherTok{\textless{}{-}} \FunctionTok{train}\NormalTok{(response }\SpecialCharTok{\textasciitilde{}}\NormalTok{ ., }\AttributeTok{data =}\NormalTok{ train\_data, }\AttributeTok{method =} \StringTok{"rf"}\NormalTok{)}
\end{Highlighting}
\end{Shaded}

\begin{verbatim}
## Warning in train.default(x, y, weights = w, ...): You are trying to do
## regression and your outcome only has two possible values Are you trying to do
## classification? If so, use a 2 level factor as your outcome column.
\end{verbatim}

\begin{verbatim}
## Warning in randomForest.default(x, y, mtry = param$mtry, ...): The response has
## five or fewer unique values.  Are you sure you want to do regression?
## Warning in randomForest.default(x, y, mtry = param$mtry, ...): The response has
## five or fewer unique values.  Are you sure you want to do regression?
## Warning in randomForest.default(x, y, mtry = param$mtry, ...): The response has
## five or fewer unique values.  Are you sure you want to do regression?
## Warning in randomForest.default(x, y, mtry = param$mtry, ...): The response has
## five or fewer unique values.  Are you sure you want to do regression?
## Warning in randomForest.default(x, y, mtry = param$mtry, ...): The response has
## five or fewer unique values.  Are you sure you want to do regression?
## Warning in randomForest.default(x, y, mtry = param$mtry, ...): The response has
## five or fewer unique values.  Are you sure you want to do regression?
## Warning in randomForest.default(x, y, mtry = param$mtry, ...): The response has
## five or fewer unique values.  Are you sure you want to do regression?
## Warning in randomForest.default(x, y, mtry = param$mtry, ...): The response has
## five or fewer unique values.  Are you sure you want to do regression?
## Warning in randomForest.default(x, y, mtry = param$mtry, ...): The response has
## five or fewer unique values.  Are you sure you want to do regression?
## Warning in randomForest.default(x, y, mtry = param$mtry, ...): The response has
## five or fewer unique values.  Are you sure you want to do regression?
## Warning in randomForest.default(x, y, mtry = param$mtry, ...): The response has
## five or fewer unique values.  Are you sure you want to do regression?
## Warning in randomForest.default(x, y, mtry = param$mtry, ...): The response has
## five or fewer unique values.  Are you sure you want to do regression?
## Warning in randomForest.default(x, y, mtry = param$mtry, ...): The response has
## five or fewer unique values.  Are you sure you want to do regression?
## Warning in randomForest.default(x, y, mtry = param$mtry, ...): The response has
## five or fewer unique values.  Are you sure you want to do regression?
## Warning in randomForest.default(x, y, mtry = param$mtry, ...): The response has
## five or fewer unique values.  Are you sure you want to do regression?
## Warning in randomForest.default(x, y, mtry = param$mtry, ...): The response has
## five or fewer unique values.  Are you sure you want to do regression?
## Warning in randomForest.default(x, y, mtry = param$mtry, ...): The response has
## five or fewer unique values.  Are you sure you want to do regression?
## Warning in randomForest.default(x, y, mtry = param$mtry, ...): The response has
## five or fewer unique values.  Are you sure you want to do regression?
## Warning in randomForest.default(x, y, mtry = param$mtry, ...): The response has
## five or fewer unique values.  Are you sure you want to do regression?
## Warning in randomForest.default(x, y, mtry = param$mtry, ...): The response has
## five or fewer unique values.  Are you sure you want to do regression?
## Warning in randomForest.default(x, y, mtry = param$mtry, ...): The response has
## five or fewer unique values.  Are you sure you want to do regression?
## Warning in randomForest.default(x, y, mtry = param$mtry, ...): The response has
## five or fewer unique values.  Are you sure you want to do regression?
## Warning in randomForest.default(x, y, mtry = param$mtry, ...): The response has
## five or fewer unique values.  Are you sure you want to do regression?
## Warning in randomForest.default(x, y, mtry = param$mtry, ...): The response has
## five or fewer unique values.  Are you sure you want to do regression?
## Warning in randomForest.default(x, y, mtry = param$mtry, ...): The response has
## five or fewer unique values.  Are you sure you want to do regression?
## Warning in randomForest.default(x, y, mtry = param$mtry, ...): The response has
## five or fewer unique values.  Are you sure you want to do regression?
## Warning in randomForest.default(x, y, mtry = param$mtry, ...): The response has
## five or fewer unique values.  Are you sure you want to do regression?
## Warning in randomForest.default(x, y, mtry = param$mtry, ...): The response has
## five or fewer unique values.  Are you sure you want to do regression?
## Warning in randomForest.default(x, y, mtry = param$mtry, ...): The response has
## five or fewer unique values.  Are you sure you want to do regression?
## Warning in randomForest.default(x, y, mtry = param$mtry, ...): The response has
## five or fewer unique values.  Are you sure you want to do regression?
## Warning in randomForest.default(x, y, mtry = param$mtry, ...): The response has
## five or fewer unique values.  Are you sure you want to do regression?
## Warning in randomForest.default(x, y, mtry = param$mtry, ...): The response has
## five or fewer unique values.  Are you sure you want to do regression?
## Warning in randomForest.default(x, y, mtry = param$mtry, ...): The response has
## five or fewer unique values.  Are you sure you want to do regression?
## Warning in randomForest.default(x, y, mtry = param$mtry, ...): The response has
## five or fewer unique values.  Are you sure you want to do regression?
## Warning in randomForest.default(x, y, mtry = param$mtry, ...): The response has
## five or fewer unique values.  Are you sure you want to do regression?
## Warning in randomForest.default(x, y, mtry = param$mtry, ...): The response has
## five or fewer unique values.  Are you sure you want to do regression?
## Warning in randomForest.default(x, y, mtry = param$mtry, ...): The response has
## five or fewer unique values.  Are you sure you want to do regression?
## Warning in randomForest.default(x, y, mtry = param$mtry, ...): The response has
## five or fewer unique values.  Are you sure you want to do regression?
## Warning in randomForest.default(x, y, mtry = param$mtry, ...): The response has
## five or fewer unique values.  Are you sure you want to do regression?
## Warning in randomForest.default(x, y, mtry = param$mtry, ...): The response has
## five or fewer unique values.  Are you sure you want to do regression?
## Warning in randomForest.default(x, y, mtry = param$mtry, ...): The response has
## five or fewer unique values.  Are you sure you want to do regression?
## Warning in randomForest.default(x, y, mtry = param$mtry, ...): The response has
## five or fewer unique values.  Are you sure you want to do regression?
## Warning in randomForest.default(x, y, mtry = param$mtry, ...): The response has
## five or fewer unique values.  Are you sure you want to do regression?
## Warning in randomForest.default(x, y, mtry = param$mtry, ...): The response has
## five or fewer unique values.  Are you sure you want to do regression?
## Warning in randomForest.default(x, y, mtry = param$mtry, ...): The response has
## five or fewer unique values.  Are you sure you want to do regression?
## Warning in randomForest.default(x, y, mtry = param$mtry, ...): The response has
## five or fewer unique values.  Are you sure you want to do regression?
## Warning in randomForest.default(x, y, mtry = param$mtry, ...): The response has
## five or fewer unique values.  Are you sure you want to do regression?
## Warning in randomForest.default(x, y, mtry = param$mtry, ...): The response has
## five or fewer unique values.  Are you sure you want to do regression?
## Warning in randomForest.default(x, y, mtry = param$mtry, ...): The response has
## five or fewer unique values.  Are you sure you want to do regression?
## Warning in randomForest.default(x, y, mtry = param$mtry, ...): The response has
## five or fewer unique values.  Are you sure you want to do regression?
## Warning in randomForest.default(x, y, mtry = param$mtry, ...): The response has
## five or fewer unique values.  Are you sure you want to do regression?
## Warning in randomForest.default(x, y, mtry = param$mtry, ...): The response has
## five or fewer unique values.  Are you sure you want to do regression?
## Warning in randomForest.default(x, y, mtry = param$mtry, ...): The response has
## five or fewer unique values.  Are you sure you want to do regression?
## Warning in randomForest.default(x, y, mtry = param$mtry, ...): The response has
## five or fewer unique values.  Are you sure you want to do regression?
## Warning in randomForest.default(x, y, mtry = param$mtry, ...): The response has
## five or fewer unique values.  Are you sure you want to do regression?
## Warning in randomForest.default(x, y, mtry = param$mtry, ...): The response has
## five or fewer unique values.  Are you sure you want to do regression?
## Warning in randomForest.default(x, y, mtry = param$mtry, ...): The response has
## five or fewer unique values.  Are you sure you want to do regression?
## Warning in randomForest.default(x, y, mtry = param$mtry, ...): The response has
## five or fewer unique values.  Are you sure you want to do regression?
## Warning in randomForest.default(x, y, mtry = param$mtry, ...): The response has
## five or fewer unique values.  Are you sure you want to do regression?
## Warning in randomForest.default(x, y, mtry = param$mtry, ...): The response has
## five or fewer unique values.  Are you sure you want to do regression?
## Warning in randomForest.default(x, y, mtry = param$mtry, ...): The response has
## five or fewer unique values.  Are you sure you want to do regression?
## Warning in randomForest.default(x, y, mtry = param$mtry, ...): The response has
## five or fewer unique values.  Are you sure you want to do regression?
## Warning in randomForest.default(x, y, mtry = param$mtry, ...): The response has
## five or fewer unique values.  Are you sure you want to do regression?
## Warning in randomForest.default(x, y, mtry = param$mtry, ...): The response has
## five or fewer unique values.  Are you sure you want to do regression?
## Warning in randomForest.default(x, y, mtry = param$mtry, ...): The response has
## five or fewer unique values.  Are you sure you want to do regression?
## Warning in randomForest.default(x, y, mtry = param$mtry, ...): The response has
## five or fewer unique values.  Are you sure you want to do regression?
## Warning in randomForest.default(x, y, mtry = param$mtry, ...): The response has
## five or fewer unique values.  Are you sure you want to do regression?
## Warning in randomForest.default(x, y, mtry = param$mtry, ...): The response has
## five or fewer unique values.  Are you sure you want to do regression?
## Warning in randomForest.default(x, y, mtry = param$mtry, ...): The response has
## five or fewer unique values.  Are you sure you want to do regression?
## Warning in randomForest.default(x, y, mtry = param$mtry, ...): The response has
## five or fewer unique values.  Are you sure you want to do regression?
## Warning in randomForest.default(x, y, mtry = param$mtry, ...): The response has
## five or fewer unique values.  Are you sure you want to do regression?
## Warning in randomForest.default(x, y, mtry = param$mtry, ...): The response has
## five or fewer unique values.  Are you sure you want to do regression?
## Warning in randomForest.default(x, y, mtry = param$mtry, ...): The response has
## five or fewer unique values.  Are you sure you want to do regression?
## Warning in randomForest.default(x, y, mtry = param$mtry, ...): The response has
## five or fewer unique values.  Are you sure you want to do regression?
## Warning in randomForest.default(x, y, mtry = param$mtry, ...): The response has
## five or fewer unique values.  Are you sure you want to do regression?
## Warning in randomForest.default(x, y, mtry = param$mtry, ...): The response has
## five or fewer unique values.  Are you sure you want to do regression?
\end{verbatim}

\begin{Shaded}
\begin{Highlighting}[]
\NormalTok{importance }\OtherTok{\textless{}{-}} \FunctionTok{varImp}\NormalTok{(modele\_foret)}
\FunctionTok{plot}\NormalTok{(importance, }\AttributeTok{main =} \StringTok{"Importance des variables"}\NormalTok{)}
\end{Highlighting}
\end{Shaded}

\includegraphics{formR_files/figure-latex/unnamed-chunk-1-20.pdf}

\end{document}
